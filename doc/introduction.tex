\chapter{Introduction}

\code{djehuty} is the data repository system developed by and for
4TU.ResearchData.  The name finds its inspiration in
\href{https://en.wikipedia.org/wiki/Thoth}{Thoth}, the Egyptian
entity that introduced the idea of writing.

\section{Obtaining the source code}
\label{sec:obtaining-tarball}

  \begin{sloppypar}
  The source code can be downloaded at the
  \href{https://github.com/4TUResearchData/djehuty/releases}%
  {Releases}%
  \footnote{\url{https://github.com/4TUResearchData/djehuty/releases}}
  page.  Make sure to download the {\fontfamily{\ttdefault}\selectfont
    djehuty-\djehutyversion{}.tar.gz} file.
  \end{sloppypar}

  Or, directly download the tarball using the command-line:

\begin{lstlisting}[language=bash]
curl -LO https://github.com/4TUResearchData/djehuty/releases/\
download/(@*\djehutyversion{}*@)/djehuty-(@*\djehutyversion{}*@).tar.gz
\end{lstlisting}

  After obtaining the tarball, it can be unpacked using the \t{tar}
  command:

\begin{lstlisting}[language=bash]
tar zxvf djehuty-(@*\djehutyversion{}*@).tar.gz
\end{lstlisting}

\section{Installing the prerequisites}
\label{sec:prerequisites}

  The \code{djehuty} program needs Python (version 3.6 or higher) and
  Git to be installed.  Additionally, a couple of Python packages need
  to be installed.  The following sections describe installing the
  prerequisites on various GNU/Linux distributions.  To put the software in
  the context of its environment, figure \ref{fig:references-graph} displays
  the complete run-time dependencies from \t{djehuty} to \t{glibc}.

  \includefigure{references-graph}{Run-time references when constructed with
    the packages from GNU Guix.}

  The web service of \code{djehuty} stores its information in a SPARQL 1.1
  \citep{sparql-11} endpoint.  We recommend either
  \href{https://blazegraph.com/}{Blazegraph}%
  \footnote{https://blazegraph.com/}
  or \href{http://vos.openlinksw.com/owiki/wiki/VOS}%
  {Virtuoso open-source edition}%
  \footnote{http://vos.openlinksw.com/owiki/wiki/VOS}.

\subsection{Installation on Enterprise Linux 7+}

  The Python packages on Enterprise Linux version 7 or higher seem to
  be too far out of date.  So installing the prerequisites involves
  two steps.

  The first step involves installing system-wide packages for Python and Git.

\begin{lstlisting}[language=bash]
yum install python39 git
\end{lstlisting}

  The second step involves using Python's \t{venv} module to
  install the Python packages in a virtual environment:

\begin{lstlisting}[language=bash]
python3.9 -m venv djehuty-env
. djehuty-env/bin/activate
cd /path/to/the/repository/checkout/root
pip install -r requirements.txt
\end{lstlisting}

\section{Installation instructions}

  After obtaining the source code (see section \refer{sec:obtaining-tarball})
  and installing the required tools (see section \refer{sec:prerequisites}),
  building involves running the following commands:

\begin{lstlisting}[language=bash]
cd djehuty-(@*\djehutyversion{}*@)
autoreconf -vif # Only needed if the "./configure" step does not work.
./configure
make
make install
\end{lstlisting}

  To run the \t{make install} command, super user privileges may be
  required.  Specify a \t{-{}-prefix} to the \t{configure}
  script to install the tools to a user-writeable location to avoid
  needing super user privileges.

  After installation, the \t{djehuty} program will be available.
