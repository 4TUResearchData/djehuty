\chapter{Knowledge graph}

  \code{djehuty} processes its information using the Resource Description
  Framework (\cite{lassila-99-rdf}).  This chapter describes the parts that
  make up the data model of \t{djehuty}.

\section{Use of vocabularies}

  Throughout this chapter, abbreviated references to ontologies are used.
  Table \ref{table:vocabularies} lists these abbreviations.

  \hypersetup{urlcolor=black}
  \begin{table}[H]
    \begin{tabularx}{\textwidth}{*{1}{!{\VRule[-1pt]}l}!{\VRule[-1pt]}X}
      \headrow
      \b{Abbreviation} & \b{Ontology URI}\\
      \evenrow
      \t{djht}         & \dhref{https://ontologies.data.4tu.nl/djehuty/\djehutyversion/}\\
      \oddrow
      \t{rdf}          & \dhref{http://www.w3.org/1999/02/22-rdf-syntax-ns\#}\\
      \evenrow
      \t{rdfs}         & \dhref{http://www.w3.org/2000/01/rdf-schema\#}\\
      \oddrow
      \t{xsd}          & \dhref{http://www.w3.org/2001/XMLSchema\#}\\
    \end{tabularx}
    \caption{\small Lookup table for vocabulary URIs and their abbreviations.}
    \label{table:vocabularies}
  \end{table}
  \hypersetup{urlcolor=LinkGray}

\section{Notational shortcuts}

  In addition to abbreviating ontologies with their prefix we use another
  notational shortcut.  To effectively communicate the structure of the RDF
  graph used by \t{djehuty} we introduce a couple of shorthand notations.

\subsection{Notation for typed triples}

  When the \code{object} in a triple is \i{typed}, we introduce the shorthand
  to only show the type, rather than the actual value of the \code{object}.
  Figure \ref{fig:typed-notation} displays this for URIs, and figure
  \ref{fig:typed-literals-notation} displays this for literals.

  \includefigure{typed-notation}{Shorthand notation for triples with an
    \code{rdf:type} which features a hollow predicate arrow and a colored
    type specifier with rounded corners.}

  Literals are depicted by rectangles (with sharp edges) in contrast to URIs
  which are depicted as rectangles with rounded edges.

  \includefigure{typed-literals-notation}{Shorthand notation for triples with
    a literal, which features a hollow predicate arrow and a colored
    rectangular type specifier.}

  When the subject of a triple is the shorthand type, assume the subject is not
  the type itself but the subject which has that type.

\subsection{Notation for \code{rdf:List}}

  To preserve the order in which lists were formed, the data model makes use
  of \code{rdf:List} with numeric indexes.  This pattern will be abbreviated
  in the remainder of the figures as displayed in figure
  \ref{fig:rdf-list-abbrev}.

  \includefigure{rdf-list-abbrev}{Shorthand notation for \code{rdf:List}
    with numeric indexes, which features a hollow double-arrow.  Lists have
    arbitrary lengths, and the numeric indexes use 1-based indexing.}

  The hollow double-arrow depicts the use of an \code{rdf:List} with numeric
  indexes.
