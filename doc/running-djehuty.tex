\chapter{Running \code{djehuty}}

Now that \code{djehuty} is installed, it's a good moment to look into its
run-time configuration options.  All configuration can be done through a
configuration file, for which an example is available at
\file{etc/djehuty/djehuty-example-config.xml}.

\section{The configuration file}

\subsection{Essential options}

\begin{tabular}{p{0.32\textwidth} p{0.62\textwidth}}
  \ifdefined\HCode
  \textbf{Option}            & \textbf{Description}\\
  \fi
  \t{bind-address}           & The address to bind a TCP socket on.\\
  \t{port}                   & The port to bind a TCK socket on.\\
  \t{base-url}               & The URL on which the instance will be available
                               to the outside world.\\
  \t{allow-crawlers}         & Set to 1 to allow crawlers in the \t{robots.txt},
                               otherwise set to 0.\\
  \t{production}             & Performs extra checks before starting. Enable
                               this when running a production instance.\\
  \t{live-reload}            & When set to 1, it reloads Python code on-the-fly.
                               We recommend to set it to 0 when running in
                               production.\\
  \t{debug-mode}             & When set to 1, it will display backtraces and
                               error messages in the web browser. When set to 0,
                               it will only show backtraces and error messages
                               in the web browser.\\
  \t{use-x-forwarded-for}    & When running \t{djehuty} behind a reverse-proxy
                               server, use the HTTP header \t{X-Forwarded-For}
                               to log IP address information.\\
  \t{disable-collaboration}  & When set to 1, it disables the ``collaborators''
                               feature.\\
  \t{enable-query-audit-log} & When set to 1, it writes every SPARQL query that
                               modifies the database in the web logs.  This can
                               be replayed to reconstruct the database at a
                               later time.  Setting this option to 0 disables
                               this feature.\\
  \t{disable-2fa}            & Accounts with privileges receive a code by e-mail
                               as a second factor when logging in.  Setting this
                               option to 1 disables the second factor
                               authentication.\\
  \t{sandbox-message}        & Display a message on the top of every page.\\
  \t{notice-message}         & Display a message on the main page.\\
  \t{maintenance-mode}       & When set to 1, all HTTP requests result in the
                               displayment of a maintenance message. Use this
                               option while backing up the database, or when
                               performing major updates.\\
\end{tabular}

\subsection{Configuring the Database}

  The \t{djehuty} program stores its state in a SPARQL 1.1 compliant
  RDF store.  Configuring the connection details is done in the
  \t{rdf-store} node.

\begin{tabular}{p{0.32\textwidth} p{0.62\textwidth}}
  \ifdefined\HCode
  \textbf{Option}            & \textbf{Description}\\
  \fi
  \t{state-graph}            & The graph name to store triplets in.\\
  \t{sparql-uri}             & The URI at which the SPARQL 1.1 endpoint can
                               be reached.\newline\newline
                               When the \t{sparql-uri} begins with \t{bdb://},
                               followed by a path to a filesystem directory,
                               it will use the BerkeleyDB back-end, for which
                               the \code{berkeleydb} Python package needs to
                               be installed.\\
  \t{sparql-update-uri}      & The URI at which the SPARQL 1.1 Update endpoint
                               can be reached (in case it is different from
                               the \t{sparql-uri}.\\
\end{tabular}

\subsection{Configuring storage}

  Storage locations can be configured with the \t{storage} node.
  When configuring multiple locations, \t{djehuty} attempts to find a
  file by looking at the first configured location, and in case it cannot
  find the file there, it will look at the second configured location,
  and so on, until it has tried each storage location.

  This allows for moving files between storage systems transparently
  without requiring specific interactions with \t{djehuty} other than
  having the files made available as a POSIX filesystem.

  One use-case that suits this mechanism is letting uploads write to
  fast online storage and later move the uploaded files to a slower but
  less costly storage.

\begin{tabular}{p{0.32\textwidth} p{0.62\textwidth}}
  \ifdefined\HCode
  \textbf{Option}            & \textbf{Description}\\
  \fi
  \t{location}               & A filesystem path to where files are stored.
                               This is a repeatable property.\\
\end{tabular}

\subsection{Customizing looks}

  With the following options, the instance can be branded as necessary.

\begin{tabular}{p{0.32\textwidth} p{0.62\textwidth}}
  \ifdefined\HCode
  \textbf{Option}             & \textbf{Description}\\
  \fi
  \t{site-name}               & Name for the instance used in the title of a
                                browser window.\\
  \t{site-description}        & Description used as a meta-tag in the HTML
                                output.\\
  \t{site-shorttag}           & Used as keyword and as Git remote name.\\
  \t{support-email-address}   & E-mail address used in e-mails sent to users
                                in automated messages.\\
  \t{custom-logo-path}        & Path to a PNG image file that will be used as
                                logo on the website.\\
  \t{custom-favicon-path}     & Path to an ICO file that will be used as
                                favicon.\\
  \t{small-footer}            & HTML that will be used as footer for all
                                pages except for the main page.\\
  \t{large-footer}            & HTML that will be used as footer on the
                                main page.\\
  \t{show-portal-summary}     & When set to 1, it shows the repository summary
                                of number of datasets, authors, collections,
                                files and bytes on the main page.\\
  \t{show-institutions}       & When set to 1, it shows the list of
                                institutions on the main page.\\
  \t{show-science-categories} & When set to 1, it shows the subjects
                                (categories) on the main page.\\
  \t{show-latest-datasets}    & When set to 1, it shows the list of latest
                                published datasets on the main page.\\
  \t{colors}                  & Colors used in the HTML output. See section
                                \ref{sec:customize-colors}.\\
\end{tabular}

\subsubsection{Customizing colors}
\label{sec:customize-colors}

  The following options can be configured in the \t{colors} section.

\begin{tabular}{p{0.32\textwidth} p{0.62\textwidth}}
  \ifdefined\HCode
  \textbf{Option}              & \textbf{Description}\\
  \fi
  \t{primary-color}            & The main background color to use.\\
  \t{primary-foreground-color} & The main foreground color to use.\\
  \t{primary-color-hover}      & Color to use when hovering a link.\\
  \t{primary-color-active}     & Color to use when a link is clicked.\\
  \t{privilege-button-color}   & The background color of buttons for
                                 privileged actions.\\
  \t{footer-background-color}  & Color to use in the footer.\\
\end{tabular}
