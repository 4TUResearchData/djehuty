\chapter{Application Programming Interface}

  The application programming interface (API) provided by \t{djehuty} allows
  for automating tasks otherwise done through the user interface.  In addition
  to automation, the API can also be used to gather additional information,
  like statistics on Git repositories.

  Throughout this chapter we provide examples for using the API using \t{curl} and \t{jq}.
  Another way of seeing the API in action is to use the developer tools in a web
  browser while performing the desired action using the web user interface.

\section{The \t{/v2} public interface}

  The \t{v2} API was designed by Figshare\footnote{\dhref{https://figshare.com}}.
  \t{djehuty} implements a backward-compatible version of it, with the
  following differences:
  \begin{enumerate}
    \item{The \t{id} property is superseded by the \t{uuid} property.}
    \item{Error handling is done through precise HTTP error codes,
        rather than always returning \t{400} on a usage error.}
  \end{enumerate}

  Unless specified otherwise, the HTTP \t{Content-Type} to interact
  with the API is \t{application/json}.  In the case an API call returns
  information, don't forget to set the HTTP \t{Accept} header appropriately.

\subsection{\t{/v2/articles} (HTTP \code{GET})}
\label{sec:v2-articles}

  The following parameters can be used:

\begin{tabular}{p{0.20\textwidth} p{0.20\textwidth} p{0.59\textwidth}}
  \ifdefined\HCode
  \textbf{Parameter}   & \textbf{Required} & \textbf{Description}\\
  \fi
  \t{order}            & Optional & Field to use for sorting.\\
  \t{order\_direction} & Optional & Can be either \code{asc} or \code{desc}.\\
  \t{institution}      & Optional & The institution identifier to filter on.\\
  \t{published\_since} & Optional & When set, datasets published before this
                                    timestamp are dropped from the results.\\
  \t{modified\_since}  & Optional & When set, only datasets modified after
                                    this timestamp are shown from the results.\\
  \t{group}            & Optional & The group identifier to filter on.\\
  \t{resource\_doi}    & Optional & The DOI of the associated journal publication.
                                    When set, only returns datasets associated
                                    with this DOI.\\
  \t{item\_type}       & Optional & Either \code{3} for datasets or \code{9}
                                    for software.\\
  \t{doi}              & Optional & The DOI of the dataset to search for.\\
  \t{handle}           & Optional & Unused.\\
  \t{page}             & Optional & The page number used in combination with
                                    \t{page\_size}.\\
  \t{page\_size}       & Optional & The number of datasets per page.  Used
                                    in combination with \t{page}.\\
  \t{limit}            & Optional & The maximum number of datasets to output.
                                    Used together with \t{offset}.\\
  \t{offset}           & Optional & The number of datasets to skip in the
                                    output.  Used together with \t{limit}.\\
\end{tabular}

  Example usage:
\begin{lstlisting}[language=bash]
curl "(@*\djehutybaseurl{}*@)/v2/articles?limit=100&published_since=2024-07-25" | jq
\end{lstlisting}

  Output of the example:
\begin{lstlisting}[language=JSON]
[ /* Example output has been shortened. */
  {
    "id": null,
    "uuid": "4f8a9423-83fc-4263-9bb7-2aa83d73865d",
    "title": "Measurement data of a Low Speed Field Test of Tractor Se...",
    "doi": "10.4121/4f8a9423-83fc-4263-9bb7-2aa83d73865d.v1",
    "handle": null,
    "url": "(@*\djehutybaseurl{}*@)/v2/articles/4f8a...865d",
    "published_date": "2024-07-26T10:39:57",
    "thumb": null,
    "defined_type": 3,
    "defined_type_name": "dataset",
    "group_id": 28589,
    "url_private_api": "(@*\djehutybaseurl{}*@)/v2/account/articles/4f8a...865d",
    "url_public_api": "(@*\djehutybaseurl{}*@)/v2/articles/4f8a...865d",
    "url_private_html": "(@*\djehutybaseurl{}*@)/my/datasets/4f8a...865d/edit",
    "url_public_html": "(@*\djehutybaseurl{}*@)/datasets/4f8a...865d/1",
    ...
  }
]
\end{lstlisting}

\subsection{\t{/v2/articles/search} (HTTP \code{POST})}

  In addition to the parameters of section \refer{sec:v2-articles}, the
  following parameters can be used.

\begin{tabular}{p{0.20\textwidth} p{0.20\textwidth} p{0.59\textwidth}}
  \ifdefined\HCode
  \textbf{Parameter}   & \textbf{Required} & \textbf{Description}\\
  \fi
  \t{search\_for}       & Optional & The terms to search for.\\
\end{tabular}

  Example usage:
\begin{lstlisting}[language=bash]
curl --request POST\
     --header "Content-Type: application/json"\
     --data '{ "search_for": "djehuty" }'\
     (@*\djehutybaseurl{}*@)/v2/articles/search | jq
\end{lstlisting}

  Output of the example:
\begin{lstlisting}[language=JSON]
[ /* Example output has been shortened. */
  {
    "id": null,
    "uuid": "342efadc-66f8-4e9b-9d27-da7b28b849d2",
    "title": "Source code of the 4TU.ResearchData repository",
    "doi": "10.4121/342efadc-66f8-4e9b-9d27-da7b28b849d2.v1",
    "handle": null,
    "url": "(@*\djehutybaseurl{}*@)/v2/articles/342e...49d2",
    "published_date": "2023-03-20T11:29:10",
    "thumb": null,
    "defined_type": 9,
    "defined_type_name": "software",
    "group_id": 28586,
    "url_private_api": "(@*\djehutybaseurl{}*@)/v2/account/articles/342e...49d2",
    "url_public_api": "(@*\djehutybaseurl{}*@)/v2/articles/342e...49d2",
    "url_private_html": "(@*\djehutybaseurl{}*@)/my/datasets/342e...49d2/edit",
    "url_public_html": "(@*\djehutybaseurl{}*@)/datasets/342e...49d2/1",
    ...
  }
]
\end{lstlisting}

\subsection{\t{/v2/articles/<dataset-id>} (HTTP \code{GET})}
\label{sec:v2-articles-dataset-id}
  Example usage:
\begin{lstlisting}[language=bash]
curl (@*\djehutybaseurl{}*@)/v2/articles/342efadc-66f8-4e9b-9d27-da7b28b849d2 | jq
\end{lstlisting}

  Output of the example:
\begin{lstlisting}[language=JSON]
{ /* Example output has been shortened. */
  "files": ...,
  "custom_fields": ...,
  "authors": ...,
  "description": "<p>This dataset contains the source code of the 4TU...",
  "license": ...,
  "tags": ...,
  "categories": ...,
  "references": ...,
  "id": null,
  "uuid": "342efadc-66f8-4e9b-9d27-da7b28b849d2",
  "title": "Source code of the 4TU.ResearchData repository",
  "doi": "10.4121/342efadc-66f8-4e9b-9d27-da7b28b849d2.v1",
  "url": "(@*\djehutybaseurl{}*@)/v2/articles/342e...49d2",
  "published_date": "2023-03-20T11:29:10",
  "timeline": ...,
  ...
}
\end{lstlisting}

\subsection{\t{/v2/articles/<dataset-id>/versions} (HTTP \code{GET})}

  Example usage:
\begin{lstlisting}[language=bash]
curl (@*\djehutybaseurl{}*@)/v2/articles/342efadc-66f8-4e9b-9d27-da7b28b849d2/versions | jq
\end{lstlisting}

  Output of the example:
\begin{lstlisting}[language=JSON]
[
  {
    "version": 1,
    "url": "(@*\djehutybaseurl{}*@)/v2/articles/342efadc-66f8-4e9b-9d27-da7b28b849d2/versions/1"
  }
]
\end{lstlisting}

\subsection{\t{/v2/articles/<dataset-id>/versions/<version>} (HTTP \code{GET})}

  Example usage:
\begin{lstlisting}[language=bash]
curl (@*\djehutybaseurl{}*@)/v2/articles/342efadc-66f8-4e9b-9d27-da7b28b849d2/versions/1 | jq
\end{lstlisting}

  The output of the example is identical to the example output of section
  \refer{sec:v2-articles-dataset-id}.

\subsection{\t{/v2/articles/<dataset-id>/versions/<version>/embargo} (HTTP \code{GET})}

  Example usage:
\begin{lstlisting}[language=bash]
curl (@*\djehutybaseurl{}*@)/v2/articles/c1274889-b797-43bd-a3b1-ee0611d58fd7/versions/2/embargo | jq
\end{lstlisting}

  Output of the example:
\begin{lstlisting}[language=JSON]
{
  "is_embargoed": true,
  "embargo_date": "2039-06-30",
  "embargo_type": "article",
  "embargo_title": "Under embargo",
  "embargo_reason": "<p>Need consent to publish the data</p>",
  "embargo_options": []
}
\end{lstlisting}


%\subsection{\t{/v2/articles/<dataset-id>}/versions/<version>/confidentiality (HTTP \code{GET})}
%\subsection{\t{/v2/articles/<dataset-id>}/versions/<version>/update\_thumb}
\subsection{\t{/v2/articles/<dataset-id>/files} (HTTP \code{GET})}


  Example usage:
\begin{lstlisting}[language=bash]
curl (@*\djehutybaseurl{}*@)/v2/articles/342efadc-66f8-4e9b-9d27-da7b28b849d2/files
\end{lstlisting}

  Output of the example:
\begin{lstlisting}[language=JSON]
[ /* Example output has been shortened. */
  {
    "id": null,
    "uuid": "d3e1c325-7fa9-4cb9-884e-0b9cd2059292",
    "name": "djehuty-0.0.1.tar.gz",
    "size": 3713709,
    "is_link_only": false,
    "is_incomplete": false,
    "download_url": "(@*\djehutybaseurl{}*@)/file/342e...49d2/d3e1...9292",
    "supplied_md5": null,
    "computed_md5": "910e9b0f79a0af548f59b3d8a56c3bf4"
  }
]
\end{lstlisting}

\subsection{\t{/v2/articles/<dataset-id>/files/<file-id>} (HTTP \code{GET})}

  Example usage:
\begin{lstlisting}[language=bash]
curl (@*\djehutybaseurl{}*@)/v2/articles/342e...49d2/files/d3e1...9292 | jq
\end{lstlisting}

  Output of the example:
\begin{lstlisting}[language=JSON]
{ /* Example output has been shortened. */
  "id": null,
  "uuid": "d3e1c325-7fa9-4cb9-884e-0b9cd2059292",
  "name": "djehuty-0.0.1.tar.gz",
  "size": 3713709,
  "is_link_only": false,
  "is_incomplete": false,
  "download_url": "(@*\djehutybaseurl{}*@)/file/342e...49d2/d3e1...9292",
  "supplied_md5": null,
  "computed_md5": "910e9b0f79a0af548f59b3d8a56c3bf4"
}
\end{lstlisting}

\subsection{\t{/v2/collections} (HTTP \code{GET})}
\label{sec:v2-collections}

  The following parameters can be used:

\begin{tabular}{p{0.20\textwidth} p{0.20\textwidth} p{0.59\textwidth}}
  \ifdefined\HCode
  \textbf{Parameter}   & \textbf{Required} & \textbf{Description}\\
  \fi
  \t{order}            & Optional & Field to use for sorting.\\
  \t{order\_direction} & Optional & Can be either \code{asc} or \code{desc}.\\
  \t{institution}      & Optional & The institution identifier to filter on.\\
  \t{published\_since} & Optional & When set, collections published before this
                                    timestamp are dropped from the results.\\
  \t{modified\_since}  & Optional & When set, only collections modified after
                                    this timestamp are shown from the results.\\
  \t{group}            & Optional & The group identifier to filter on.\\
  \t{resource\_doi}    & Optional & The DOI of the associated journal publication.
                                    When set, only returns collections associated
                                    with this DOI.\\
  \t{doi}              & Optional & The DOI of the collection to search for.\\
  \t{handle}           & Optional & Unused.\\
  \t{page}             & Optional & The page number used in combination with
                                    \t{page\_size}.\\
  \t{page\_size}       & Optional & The number of collections per page.  Used
                                    in combination with \t{page}.\\
  \t{limit}            & Optional & The maximum number of collections to output.
                                    Used together with \t{offset}.\\
  \t{offset}           & Optional & The number of collections to skip in the
                                    output.  Used together with \t{limit}.\\
\end{tabular}

  Example usage:
\begin{lstlisting}[language=bash]
curl "(@*\djehutybaseurl{}*@)/v2/collections?limit=100&published_since=2024-07-25" | jq
\end{lstlisting}

Output of the example:
\begin{lstlisting}[language=JSON]
[ /* Example output has been shortened. */
  {
    "id": null,
    "uuid": "0fe9ab80-6e6a-4087-a509-ce09dddfa3d9",
    "title": "PhD research 'Untangling the complexity of local water ...'",
    "doi": "10.4121/0fe9ab80-6e6a-4087-a509-ce09dddfa3d9.v1",
    "handle": "",
    "url": "(@*\djehutybaseurl{}*@)/v2/collections/0fe9...fa3d9",
    "timeline": {
      "posted": "2024-08-13T14:09:52",
      "firstOnline": "2024-08-13T14:09:51",
      ...
    },
    "published_date": "2024-08-13T14:09:52"
  },
  ...
]
\end{lstlisting}

\subsection{\t{/v2/collections/search} (HTTP \code{POST})}

  In addition to the parameters of section \refer{sec:v2-collections}, the
  following parameters can be used.

\begin{tabular}{p{0.20\textwidth} p{0.20\textwidth} p{0.59\textwidth}}
  \ifdefined\HCode
  \textbf{Parameter}   & \textbf{Required} & \textbf{Description}\\
  \fi
  \t{search\_for}      & Optional          & The terms to search for.\\
\end{tabular}

  Example usage:
\begin{lstlisting}[language=bash]
curl --request POST\
     --header "Content-Type: application/json"\
     --data '{ "search_for": "wingtips" }'\
     (@*\djehutybaseurl{}*@)/v2/collections/search | jq
\end{lstlisting}

  Output of the example:
\begin{lstlisting}[language=JSON]
[  /* Example output has been shortened. */
  {
    "id": 6070238,
    "uuid": "3dfc4ef2-7f79-4d33-81a7-9c6ae09a2782",
    "title": "Flared Folding Wingtips - TU Delft",
    "doi": "10.4121/c.6070238.v1",
    "handle": "",
    "url": "(@*\djehutybaseurl{}*@)/v2/collections/3dfc...2782",
    "timeline": {
      "posted": "2023-04-05T15:05:04",
      "firstOnline": "2023-04-05T15:05:03",
      ...
    },
    "published_date": "2023-04-05T15:05:04"
  },
  ...
]
\end{lstlisting}

\subsection{\t{/v2/collections/<collection-id>} (HTTP \code{GET})}

  Example usage:
\begin{lstlisting}[language=bash]
curl (@*\djehutybaseurl{}*@)/v2/collections/3dfc4ef2-7f79-4d33-81a7-9c6ae09a2782 | jq
\end{lstlisting}

  Output of the example:
\begin{lstlisting}[language=JSON]
{ /* Example output has been shortened. */
  "version": 3,
  ...
  "description": "<p>This collection contains the results of the work ...",
  "categories": [ ... ],
  "references": [],
  "tags": [ ... ],
  "created_date": "2024-08-08T15:48:55",
  "modified_date": "2024-08-12T11:24:39",
  "id": 6070238,
  "uuid": "3dfc4ef2-7f79-4d33-81a7-9c6ae09a2782",
  "title": "Flared Folding Wingtips - TU Delft",
  "doi": "10.4121/c.6070238.v3",
  "published_date": "2024-08-12T11:24:40",
  "timeline": ...
  ...
}
\end{lstlisting}

\subsection{\t{/v2/collections/<collection-id>/versions} (HTTP \code{GET})}

  Example usage:
\begin{lstlisting}[language=bash]
curl (@*\djehutybaseurl{}*@)/v2/collections/3dfc4ef2-7f79-4d33-81a7-9c6ae09a2782/versions | jq
\end{lstlisting}

  Output of the example:
\begin{lstlisting}[language=JSON]
[ /* Example output has been shortened. */
  {
    "version": 3,
    "url": "(@*\djehutybaseurl{}*@)/v2/collections/3dfc...2782/versions/3"
  },
  {
    "version": 2,
    "url": "(@*\djehutybaseurl{}*@)/v2/collections/3dfc...2782/versions/2"
  },
  {
    "version": 1,
    "url": "(@*\djehutybaseurl{}*@)/v2/collections/3dfc...2782/versions/1"
  }
]
\end{lstlisting}

\subsection{\t{/v2/collections/<collection-id>/versions/<version>} (HTTP \code{GET})}

  Example usage:
\begin{lstlisting}[language=bash]
curl (@*\djehutybaseurl{}*@)/v2/collections/3dfc4ef2-7f79-4d33-81a7-9c6ae09a2782/versions/2 | jq
\end{lstlisting}

  Output of the example:
\begin{lstlisting}[language=JSON]
{ /* Example output has been shortened. */
  "version": 2,
  ...
  "description": "<p>This collection contains the results of the work ...",
  "categories": [ ... ],
  "references": [],
  "tags": [ ... ],
  "references": [],
  "tags": [ ... ],
  "authors": [ ... ],
  "created_date": "2023-04-05T15:07:35",
  "modified_date": "2023-05-26T15:19:11",
  "id": 6070238,
  "uuid": "3dfc4ef2-7f79-4d33-81a7-9c6ae09a2782",
  "title": "Flared Folding Wingtips - TU Delft",
  "doi": "10.4121/c.6070238.v2",
  ...
}
\end{lstlisting}

\subsection{\t{/v2/categories} (HTTP \code{GET})}
\label{sec:v2-categories}

  Each dataset and collection is categorized using a controlled vocabulary
  of categories.  This API endpoint provides those categories.

  Example usage:
\begin{lstlisting}[language=bash]
curl (@*\djehutybaseurl{}*@)/v2/categories | jq
\end{lstlisting}

  Output of the example:
\begin{lstlisting}[language=JSON]
[ /* Example output has been shortened. */
  {
    "id": 13622,
    "uuid": "01fddd41-68d2-4e28-9d9c-18347847e7d1",
    "title": "Mining and Extraction of Energy Resources",
    "parent_id": 13620,
    "parent_uuid": "6e5bdc69-96db-41e4-ac0b-18812b46c49c",
    "path": "",
    "source_id": null,
    "taxonomy_id": null
  },
  {
    "id": 13443,
    "uuid": "026f555c-2826-4a83-97ff-0f230fb54ddb",
    "title": "Livestock Raising",
    "parent_id": 13440,
    "parent_uuid": "45a8c849-ab59-4302-af79-09b8c0677df8",
    "path": "",
    "source_id": null,
    "taxonomy_id": null
  },
  ...
]
\end{lstlisting}

\subsection{\t{/v2/licenses} (HTTP \code{GET})}
\label{sec:v2-licenses}

  Publishing a dataset involves communicating under which conditions it can be
  re-used.  The licenses under which you can publish a dataset can be found with
  this API endpoint.

  Example usage:
\begin{lstlisting}[language=bash]
curl (@*\djehutybaseurl{}*@)/v2/licenses | jq
\end{lstlisting}

  Output of the example:
\begin{lstlisting}[language=JSON]
[ /* Example output has been shortened. */
  {
    "value": 1,
    "name": "CC BY 4.0",
    "url": "https://creativecommons.org/licenses/by/4.0/",
    "type": "data"
  },
  {
    "value": 10,
    "name": "CC BY-NC 4.0",
    "url": "https://creativecommons.org/licenses/by-nc/4.0/",
    "type": "data"
  },
  ...
]
\end{lstlisting}

\section{The \t{/v2} private interface}

  The interaction with the \t{v2} private interface API requires an API token.
  Such a token can be obtained from the dashboard page after logging in.  This
  token can then be passed along in the \t{Authorization} HTTP header as:
\begin{lstlisting}
Authorization: token YOUR_TOKEN_HERE
\end{lstlisting}

\subsection{\t{/v2/account/articles} (HTTP \code{GET})}

  This API endpoint lists the draft datasets of the account to which the
  authorization token belongs.

  The following parameters can be used:

\begin{tabular}{p{0.20\textwidth} p{0.20\textwidth} p{0.59\textwidth}}
  \ifdefined\HCode
  \textbf{Parameter}   & \textbf{Required} & \textbf{Description}\\
  \fi
  \t{page}             & Optional & The page number used in combination with
                                    \t{page\_size}.\\
  \t{page\_size}       & Optional & The number of datasets per page.  Used
                                    in combination with \t{page}.\\
  \t{limit}            & Optional & The maximum number of datasets to output.
                                    Used together with \t{offset}.\\
  \t{offset}           & Optional & The number of datasets to skip in the
                                    output.  Used together with \t{limit}.\\
\end{tabular}

  Example usage:
\begin{lstlisting}[language=bash]
curl -H "Authorization: token YOUR_TOKEN_HERE" (@*\djehutybaseurl{}*@)/v2/account/articles | jq
\end{lstlisting}

  Output of the example:
\begin{lstlisting}[language=JSON]
{ /* Example output has been shortened. */
  "id": null,
  "uuid": "6ddd7a31-8ad8-4c20-95a3-e68fe716fa42",
  "title": "Example draft dataset",
  "doi": null,
  "handle": null,
  "url": "(@*\djehutybaseurl{}*@)/v2/articles/6ddd7a31-8ad8-4c20-95a3-e68fe716fa42",
  "published_date": null,
  ...
}
\end{lstlisting}

\subsection{\t{/v2/account/articles} (HTTP \code{POST})}

  This API endpoint can be used to create a new dataset.

  The following parameters can be used:

\begin{tabular}{p{0.20\textwidth} p{0.20\textwidth} p{0.59\textwidth}}
  \ifdefined\HCode
  \textbf{Parameter} & \textbf{Data type}   & \textbf{Description}\\
  \fi
  \t{title}          & \t{string}           & The title of the dataset.\\
  \t{description}    & \t{string}           & A description of the dataset.\\
  \t{tags}           & list of \t{string}s  & Keywords to enhance the
                                              findability of the dataset. Instead
                                              of using the key \code{tags}, you
                                              may also use the key
                                              \code{keywords}.\\
  \t{references}     & list of \t{string}s  & URLs to resources referring to
                                              this dataset, or resources that
                                              this dataset refers to.\\
  \t{categories}     & list of \t{string}s  & Categories are a controlled
                                              vocabulary and can be used to
                                              make the dataset findable in
                                              the categorical overviews.
                                              The \t{string} values expected
                                              here can be found under the
                                              \code{uuid} property with a
                                              call to \code{/v2/categories}.
                                              For more details, see section
                                              \refer{sec:v2-categories}.\\
  \t{authors}        & list of author records & \\
  \t{defined\_type}  & \t{string}           & One of: \t{figure},
                                              \t{online resource},
                                              \t{preprint}, \t{book},
                                              \t{conference contribution},
                                              \t{media}, \t{dataset},
                                              \t{poster},
                                              \t{journal contribution},
                                              \t{presentation},
                                              \t{thesis} or \t{software}.\\
  \t{funding}        & \t{string}           & One-liner to cite funding.\\
  \t{funding\_list}  & list of funding records & \\
  \t{license}        & \t{integer}          & Licences communicate under which
                                              conditions the dataset can be
                                              re-used.  The \t{integer} value
                                              to submit here can be found as
                                              the \t{value} property in a call
                                              to \code{/v2/licences}. For more
                                              details, see section
                                              \refer{sec:v2-licenses}.\\
  %\t{language}       & \t{string}           & \\
  \t{doi}            & \t{string}           & Do not use this field as a DOI
                                              will be automatically assigned
                                              upon publication..\\
  \t{handle}         & \t{string}           & Do not use this field as it is
                                              deprecated.\\
  \t{resource\_doi}  & \t{string}           & The URL of the DOI of an
                                              associated peer-reviewed
                                              journal publication.\\
  \t{resource\_title} & \t{string}          & The title of the associated
                                              peer-reviewed journal
                                              publication.\\
  %\t{group_id}       & \t{integer}          & \\
  \t{publisher}      & \t{string}           & The name of the data repository
                                              publishing the dataset.\\
  \t{custom\_fields} & list of key-value pairs & \\
  \t{timeline}       &                      & Do not use this field because it
                                              will be automatically populated
                                              during the publication process.\\
\end{tabular}
